%!TeX program = xelatex
\documentclass[12pt,hyperref,a4paper,UTF8]{ctexart}
\usepackage{UESTCpapers}

%%-------------------------------正文开始---------------------------%%
\begin{document}

%%-----------------------封面--------------------%%
\cover

%%------------------摘要-------------%%
%\begin{abstract}
%
%在此填写摘要内容
%
%\end{abstract}

\thispagestyle{empty} % 首页不显示页码

%%--------------------------目录页------------------------%%
\newpage
\tableofcontents

%%------------------------正文页从这里开始-------------------%
\newpage

%%可选择这里也放一个标题
%\begin{center}
%    \title{ \Huge \textbf{{标题}}}
%\end{center}
\section{模板说明}


\section{模板说明}
本模板主要适用于一些课程的平时论文以及期末论文,默认页边距为2.5cm,中文宋体,英文Times New Roman,字号为12pt(小四)。

编译方式:\verb|xelatex -> bibtex -> xelatex*2|


默认模板文件由以下四部分组成:
\begin{itemize}
    \item \texttt{main.tex} 主文件
    \item \texttt{reference.bib} 参考文献,使用bibtex
    \item \texttt{BNUpapers.sty} 文档格式控制,包括一些基础的设置,如页眉、标题、姓名等
    \item \texttt{figures} 放置图片的文件夹
\end{itemize}

第一次使用时需前往\texttt{BNUpapers.sty} 对标题、姓名、学号、院系、页眉等进行设置,设置完后即可一劳永逸,封面logo亦可替换

默认带有封面页以及目录页,页码从目录页开始

\section{一些插入功能}
\subsection{插入公式}
行内公式$v-\varepsilon+\phi=2$。

插入行间公式如\autoref{Euler}:
\begin{equation}
    v-\varepsilon+\phi=2
    \label{Euler}
\end{equation}

\subsection{插入图片}
bnu校徽如\autoref{bnu}所示,注意这里使用了\verb|~\autoref{}|命令,也就是会自动生成“图”“式”等前缀,无需手动输入。

\begin{figure}[!htbp]
    \centering
    \includegraphics[width =.4\textwidth]{figures/bnu.pdf}
    \caption{北京师范大学}
    \label{bnu}
\end{figure}

插入上面图片的代码:

\begin{verbatim}
    \begin{figure}[!htbp]
        \centering
        \includegraphics[width =.4\textwidth]{figures/bnu.pdf}
        \caption{北京师范大学}
        \label{bnu}
    \end{figure}
\end{verbatim}

\subsection{插入文本框}
本模板定义了一个圆角灰底的文本框,使用简化命令\verb|\tbox{}|即可,如果你不喜欢,可以前往 \texttt{BNUpapers.sty}对其进行修改。

\tbox{
    这是一个圆角灰底的文本框
}

\subsection{插入表格}
本模板文件如\autoref{doc}所示。
\begin{table}[!htbp]
    \centering
    \begin{tabular}{l  | l}
    \hline
        文件名 & 说明 \\
        \hline
        \texttt{main.tex}  & 主文件 \\
        \texttt{reference.bib} & 参考文献 \\
        \texttt{BNUpapers.sty}  & 文档格式控制\\
        \texttt{figures}  & 图片文件夹 \\
        \hline
    \end{tabular}
    \caption{本模板文件组成}
    \label{doc}
\end{table}

%\section{定理环境}
%\begin{Theorem}
%\end{Theorem}
%
%\begin{Lemma}
%\end{Lemma}
%
%\begin{Corollary}
%\end{Corollary}
%
%\begin{Proposition}
%\end{Proposition}
%
%\begin{Definition}
%\end{Definition}
%
%\begin{Example}
%\end{Example}
%
%\begin{proof}
%\end{proof}

\subsection{插入参考文献}
直接使用\verb|\cite{}|即可。

例如:


   \textit{ 此处引用了文献\cite{0Isaac}。此处引用了文献\cite{2016The}}


引用过的文献会自动出现在参考文献中。

\section{写在最后}
\subsection{发布地址}
\begin{itemize}
    \item Github: \url{https://github.com/LeyuDame/BNUpapers}
    \item Overleaf:  \url{https://www.overleaf.com/latex/templates/bnuke-cheng-lun-wen-mo-ban/bcwvxncqffkw}
\end{itemize}
\subsection{待实现的功能}
一般在数学中文论文中,常常用实心全角句点“.”代替句号“。”,为实现这一功能,理论上可以用\verb|xeCJK|里的\verb|fullwidth-stop|来进行字符映射,从而实现句号的转换,但在实际编译中常常因为各种原因报错,希望以后能实现这一功能。



1. asset和amount存在多重共线性,可以用lvtratio代替
2. 存在序列相关性,因为amount会影响坏账金额的多少。
3. LS TARGET C AGE ANNAMOUNT CAR FAMILY GEND GRADUATE HOUSE INCOME MARRY  POVERTY RICH
4. PUPULATION 比poverty和rich差(必须先把proverty和rich和population放在一起)
5. LS TARGET C AGE AnnaMOUNT CAR FAMILY GEND GRADUATE HOUSE INCOME MARRY  
annamount是冗余
%%----------- 参考文献 -------------------%%
%在reference.bib文件中填写参考文献,此处自动生成

\reference


\end{document}